\lang{Japanese} % Japanese or English
\studentnumber{1811066}
\doctitle{\mastersthesis}
\major{\engineering}    % 工学(\engineering) or 理学(\science) 
\title{反射音の周波数情報を用いた\\
透明・鏡平面の方位角推定手法の開発}
\ptitle{}
\etitle{Estimating Azimuth Angle of \\
Transparent Plate and Mirror \\
Based on Reflection Sound Property\\ 
in Frequency Domain}
\eptitle{}
%(姓と名の間に空白を入れて下さい)
\author{片山 寛基}
\pauthor{}
%(first name, last name の順に記入し、先頭文字のみを大文字にする。)
\eauthor{Hiroki Katayama}
\epauthor{}
%
% 論文提出年月日
%
\syear{2019}
\smonth{3}
\sday{15}
%
% プログラムの選択
%
%\department{\is}  % 情報理工学
%\department{\cb}    % 情報生命科学
\department{\cp}   % 知能社会創生科学
%\department{\ds}    % データサイエンス
%
%
\cmembers{小笠原 司 教授}{(主指導教員)}
         {浦岡 行治 教授}{(副指導教員)}
         {中村 哲 教授}{(副指導教員)}
         {高松 淳 准教授}{(副指導教員)}
\addcmembers{丁 明 助教}{(副指導教員)}
            {Gustavo Alfonso Garcia Ricardez 助教}{(副指導教員)}
            {}{}
            {}{}
\ecmembers{Professor Tsukasa Ogasawara}{(Supervisor)}
          {Professor Yukiharu Uraoka}{(Co-supervisor)}
          {Professor Satoshi Nakamura}{(Co-supervisor)}
          {Associate Professor Jun TAKAMATSU}{(Co-supervisor)}
\eaddcmembers{Assistant Professor Ding Ming}{(Co-supervisor)}
             {Assistant Professor GARCIA RICARDEZ Gustavo Alfonso}{(Co-supervisor)}
             {}{}
             {}{}
\keywords{ロボット聴覚, 物体形状認識, 透明物体認識, サポートベクター回帰, 平面波}
\pkeywords{}
\ekeywords{Robot Audition, Object Shape Recognition, Transparent Object Recognition, Support Vector Regression, Plane Wave}
\epkeywords{}
\abstract{
ロボットにおいて、暗闇での把持や、透明物体の把持するにはまだ、問題がある。解決策として、音波を用いての法線方向推定手法を提案する。Photometric Stereoに代表される陰影解析を参考にして、これまで音波では使われてこなかった法線方向推定手法を開発した。その中では、音波の特徴となる反射特性が周波数の高さによって変化することを用いている。
}
\pabstract{}
\eabstract{
 In robots, there are still problems in gripping in the dark and gripping transparent objects. As a solution, we propose a normal direction estimation method using sound waves.With reference to Shape-from-Shading represented by Photometric Stereo, we have developed a normal direction estimation method that has not been used in sound waves until now. In my method, it is used that the reflection characteristic that is a characteristic of a sound wave changes depending on the height of the frequency.
}

\epabstract{}
%%%%%%%%%%%%%%%%%%%%%%%%% document starts here %%%%%%%%%%%%%%%%%%%%%%%%%%%%