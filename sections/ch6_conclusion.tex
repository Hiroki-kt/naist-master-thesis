\section{結言}
\label{chap:conclusion}

\subsection{まとめ}
\label{summary}
本論文では,ロボットでの把持が難しい透明物体や鏡といったカメラでは認識しにくく物体に対して、音波を用いることによって、形状を認識することで解決できるのはないかと考えた。しかし、これまでの時間領域の情報では空間分解能に限界があり、形状認識が難しい。そこで、光学におけるPhotometric Stereoを参考にし、音の周波数領域の音の大きさに着目することで、法線方向を推定できるのではないかと考えた。

そこで、2通りの方法によって、まずは一番簡単な形状である平面板の方位角の推定を行った。一つ目はモデルベースで考える手法、二つ目はサポートベクター回帰を用いる手法である。

結果として、モデルベースで解く手法は音の干渉の影響により推定自体できなかった。次にサポートベクター回帰を用いた場合は、反射音をマイクで取得できる範囲(今回は方位角-45 [deg] - 45[deg)の間で、誤差5 [deg]以内で推定ができた。この誤差5 [deg]はロボットがものをつかむ際には許容できる範囲であると考えている。

また,本論文では、モデルベースで解けていない原因を考えるためにSVRではどういった部分に着目しているのか。どこまでSVRで認識が可能なのかを検証実験を行った。

結果として、・・・・

しかし、・・・・・
\subsection{今後の課題と展望}
\label{future_work}
本論文の結果について、到達時間差を用いて距離を求める超音波距離センサでも面の方位角推定は可能である。モデルベースで解くことで初めて、実際のPhotometric Stereoのように空間分解能を改善することができると考えている。今後はモデルの定式化に向けて、2点ほど解決策を考えている。

一点目は超音波を使用することである。今回は反射の法則が1000 [Hz] - 2000 [Hz]で変わることを用いるために可聴域を使用した。しかし、可聴域よりも超音波のほうが指向性が高く、より光の波の性質に近い。超音波を用いて研究を行いたいと考えている。

二点目は超音波でもピンクTSP信号を用いることである。これは[??]より、こうもりが環境を認識するために用いている音だと知られている。この信号を用いることによって、周波数領域の情報も取得できる。

以上の2点の方法について、より今後知識を深め、Photometric Stereoで使用されている。Lambertモデルといった光学の反射モデルのようなモデルをまずは音でも作成していきたいと考えています。

\newpage