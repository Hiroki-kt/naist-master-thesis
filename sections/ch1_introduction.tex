\section{緒言}
\label{chap:introduction}
\pagenumbering{arabic}

\subsection{研究背景}
\label{sec:back_ground}
家庭内には、ロボットにおいて把持が難しい物品が数多く存在する。例としては、コップやペットボトルといった透明な物体、鏡などの光が鏡面反射する物体がある。更に、夜間などの光がない環境でも認識は難しい。そのため、夜間は人間と同じで明かりがないと動くことができない。このようなロボットが不得手とする物体の位置と姿勢を認識する需要は増加している。特に、透明物体の把持については、コップやペットボトルといった身近なものにも存在するため、需要は大きい。

このような透明な物体"検出"手法に関して研究は多数存在する。使用センサとしては、光学式距離センサ\cite{url_omron}、カメラ\cite{Yichao2016transparentdetect}、超音波センサ\cite{ugajin2012ultrasoundobject}等、その種類も様々である。このような研究により、どの位置に透明な物体があるのかを認識することは可能である。

しかし、透明な物体"把持"について考えると、別の問題が出てくる。それは物体把持をする際に、位置というのは非常に重要であるが、位置だけでなく、同時に面の法線方向も重要ということである。\cite{wan2018manipulationsurfacenormal}

このような、透明物体の形状や法線方向の認識はロボットにおける透明物体の把持において重要である。

\subsection{関連研究}
\label{sec:related_works}
ロボットにおける透明物体認識の研究を紹介する。1つ目はカメラを用いる手法である。これは非接触で認識可能であるため、一般的であり、\secref{related_work_camera}にて紹介する。2つ目は触覚センサを用いる手法である。これは、カメラとともに使われることも多く、\secref{related_work_tactile}で紹介する。3つ目は音波を用いた手法である。これは、音波が非接触で、かつ物体の色の影響を受けずに測定可能である性質を利用できる。そこで、対象が透明物体だけでなく、音波を用いての物体形状認識手法を\secref{related_work_sound}で紹介する。

次に、物体把持において、重要な物体の法線方向を取得する手法について、これまでに提案された手法は,主に2つの方法が提案されてきた.1つ目は,距離センサを用いる場合である。物体面上の任意の3点の位置を計測することで、平面をあてはめることで法線方向を推定する手法がある。
2つ目はカメラを用いる場合である。Photometric Stereoに代表される陰影を用いることで法線方向を直接推定する手法がある.これら2つの手法に関する研究を\secref{related_work_surface}

これらの研究に加えて、最後に、これまで紹介した研究で用いられている一般的なセンシングデバイス(画像、音、触覚)についてまとめ、\secref{related_work_sensor}で紹介し、物体の形状認識についてそれぞれの利点や欠点などの特徴について述べる

% 非接触で物体の色の影響を受けないとなると候補として上がるのが音である。音波は物体色に対してロバストであり、かつコウモリが使用するエコーロケーションに代表されるように物体の認識に関しての研究もある。

\subsubsection{カメラを用いた透明物体認識}
\label{sec:related_work_camera}
% 例えば、コップやペットボトルといった透明な物体の認識を人間は目で行う。そのため、ロボットでもカメラを用いて透明物体を認識することはできると考えられている。しかし、現実はカメラでの透明な物体の認識は難しく、
透明物体ではないが、3DCADモデルを元にして、カメラを使い認識を行う研究がある。Jeffreyら\cite{jeffrey2017dex-net2}によって作成されたDex-Net:2.0では物体の3DCADモデルを用いることによって、物体の把持計画を立てる。このように、現在物体の3DCADモデルがあれば、カメラによって様々な物体を把持することができる。

しかし、ここで問題がある。それは、透明物体の場合、形状を3Dスキャンする技術は確立されておらず、まずどのようにして透明物体のCADモデルを作成するのか?といった問題がある\cite{fabrice2012transparent3d},\cite{Ihrke2008stateof} 。
そこで、はじめに思いつくのは手動でCADモデルを作ることで、透明物体の認識を行う手法である。しかし、この手法を用いる場合、3DCADモデルを作成するのに、時間がかかり、効率的ではないという問題点があった。そのため、Phillipsら\cite{phillips20113dcadinternet}は、インターネットから似ている3DCADモデルを探してくることによって、より簡単に認識を行った。しかし、似た形状の3DCADモデルがない場合、自分で作成する必要がある。Osadchyら\cite{osadchy20033dcadsplay}は透明なものにパウダーやスプレーを行うことによって、物体表面を不透明にし、従来の3Dスキャン技術を用いて3Dモデルを作成した。

\if 0
Yichaoら[]は、...
\fi

このような、透明物体の3DCADモデル作成手法をロボットの把持への応用した例として、Ilyaら\cite{iiya2013transparent}は、Microsoft社のKinectを用いて透明な物体の認識を行った。このとき、透明な物体の表面をスプレーで不透明にすることで、3DCADモデルを作成し、そのデータとRGB-Dカメラの距離情報を合わせることで、透明物体の物体姿勢の認識をリアルタイムに行い、透明物体の把持を行った。

\if 0
Ashutoshら[]は...

Luoら[]は...
\fi

しかし、これらの手法は多くの場合、3DCADモデルを使用する必要があるため、既知の物体のみに対応しており、未知の物体の認識は難しい。

\subsubsection{触覚センサを用いた透明物体認識}
\label{sec:related_work_tactile}
Kimら\cite{kim2018tactilesensor}は、磁気を利用した触覚センサを開発した。彼らはこの触覚センサーを指ごとに3つ使用した3指グリッパーを作成した。センサ感度は0.016mV/kPaで検出範囲は350kPaであった。そのため、この触覚フィードバックを使用することで、様々な形状の対象物体(例えば、造花、ワイングラスや紙コップ)を損傷な把持に成功した。
% しかし、触覚センサのみでは決められた位置の物体を掴むことはできても未知の位置にある物体を掴みたい際や、把持戦略を立てたい場合には難しい。(なぜ??)

Shaoxiongら\cite{wang2018tactilesensor}は、触覚を用いた形状認識は全体形状を認識する効率が低い問題点に対して、カメラによって、物体の大体の形状を事前知識より認識後、触覚センサによって未知の物体でも正確な3Dモデルをより効率的に作成した。これにより、カメラのみでは認識が難しい透明なペットボトルの3Dモデルを作成することに成功した。
% しかし、特にペットボトルの精度は悪く、難しい点である。

% このように、触覚センサを用いることによって、透明な物体でもロバストに認識ができるものの接触が必要であり、そこはカメラに頼らざるを得ないために大幅に結果が改善されるといったことはない。[接触が必要」だから何?接触してはいけないのか?落ちは何?もし本当にこれを言いたいならば、カメラで頑張って透明物体を認識する手法を開発すればいいのでは?]

このように、マルチモーダルなセンシングにより、視覚の弱点である2D-3Dの曖昧さ,触覚の弱点である形状全体を再構築するための効率の低さについて、お互いの強みによって補う合うことによって、より汎用性の高いセンシングシステムを作成するこができる。

\subsubsection{音を用いた物体認識}
\label{sec:related_work_sound}
音波を用いての物体形状認識手法としてもっとも用いられる手法として、音を発してから物体表面で反射視帰ってくるまでの時間を用いた距離計測により物体形状を認識する手法が研究されている。
Olayaら\cite{oday2018underwatersonar}は、水中における、ソナーを使用した環境認識を行った。水中の場合は空気中に比べて音速が早く、減衰も少ないため理想的な環境ではある。しかし、空気中では音の減衰の影響を考慮する必要があり、同様の手法は用いることができない。
Davidら\cite{david2019nlossoundimaging}は,死角にある物体の形状をスピーカから出した音が反射して帰ってきたときの時間差を見ることによって、見えない位置にある物体の形状を認識した。しかし、物体が40 [cm] $\times$ 40 [cm]以上のものを対象としており、空間分解能は1.5 [cm]である。

Amitら\cite{amit2017gesture}は、人間の手のジェスチャーを音を使って識別する手法を開発した。Ivanら\cite{ivan2014ultrasoundimaging}が作成した超音波マイクアレイを用いて、スピーカから出た音が反射してマイクに入る音の変化よって、ジェスチャーを分類した。このときに、到達時間差だけでなく、音波の強さの情報を用いることによって、精度向上だけでなく、到達時間差だけのときと比べて、比較的に小さい物体の形状認識を成功した。

Ohtaniら\cite{ohtani2007transparentsound}は、音波を用いて5種類の透明なペットボトルの分類を行った。超音波マイクロホンアレイを用いる事によって、スピーカから出た音が物体に反射して帰って来た際にマイクに入る音の強さの変化に対して、Deepleaningを使用することによって、ペットボトルのような軽い物体の分類に成功した。

しかし、これらの手法の多くの場合、物体姿勢や形状の分類はできているが、法線方向の数値での推定までできていない。

\subsection{法線方向推定手法}
\label{sec:related_work_surface}
距離センサを用いる方法として、
Yamaguchiら\cite{yamaguchi2019robotproximity}は、光学式の距離センサアレイの開発を行い。対象物体反射して受光器で取得した光の到達時間差だけでなく、反射光の強さも測定するセンサをエンドエフェクタの指先に敷き詰めることによって、掴む部分の物体形状を認識し、物体把持を行う手法を提案した。さらに、反射光の波の強さから布と紙パックの識別をすることで、バックの中に入っている対象物体の取り出しが可能となった。しかし、欠点としては透明物体の把持は考慮されていない。

2つ目のカメラを用いる方法として、
Photometric Stereoが挙げられる。この手法は光源の方向が移動することによって、カメラにうつる物体の陰影は変化することを利用している。この陰影の変化について光の反射モデルを当てはめることによって、写真の1ピクセルの明るさとその点の法線方向に関係があるというランバート反射モデル\cite{micheal1994diffusereflection}を利用した手法である。
Guanyingら\cite{guanying2019photometric}は、Photometric  Stereoの精度を上げるべく、Deep learning手法を付け加えることによって、物体の色や環境光に対して、これまでの手法よりもロバストな手法を提案した。

1つ目の距離センサを用いる場合の手法に関しては、透明な物体に対してロバストな音を使った手法もある。しかし、2つ目の中でも特ににのPhotometric Stereoに代表される法線方向を取得する手法は透明物体や鏡に適応されることは少なく、さらに、特に光の陰影を用いる手法は、同じ波である音波を用いて法線方向を推定する手法は我々の知る限り存在しない.

\subsubsection{センサ}
\label{sec:related_work_sensor}
これまで、透明物体の認識について、光(距離センサ、カメラ)、触覚、音の三種類のセンサの視点から関連研究を紹介した。
\tabref{related_work_summry}では、それぞれのセンサについて、特徴をまとめている。
% 上記の関連研究で物体把持まで行った研究は少ない。その理由としては、\secref{back_ground}でも述べたが、物体を把持する際には、物体位置だけでなく、法線方向も重要になる。

\begin{table}[b]
    \centering
    \caption{Characteristic of each sensor}
    \vspace{1zh}
    \begin{tabular}{c|p{12em}|p{12em}}\hline
        Sensor & Strong point & Weak point \\\hline\hline
        Optical & Measurement range is large  & Object color like transparent \\\hline
        Tactile & Mesurement the shape accurately & Inefficient to measure entire shape \\\hline
        Auditional & Robustness against object color & Spacial resolution is low \\\hline
    \end{tabular}
    \label{tab:related_work_summry}
\end{table}

\tabref{related_work_summry}より、カメラと触覚センサの組み合わせについては、\secref{related_work_tactile}で述べたが、有用である。しかし、カメラ、触覚センサのマルチモーダルセンシングシステムの弱点として、物体近傍のセンシングの弱さがある。カメラでは、物体近傍では、オクルージョンが発生することが多く、触覚センサは接触するまでは認識ができない。このような近接覚の認識手法として、小山ら\cite{koyama2015proximity2}は光学式の距離センサアレイをエンドエフェクタの指先につけ、物体表面の形状を素早く認識することによって、移動している物体を掴むことに成功している。しかし、光学式の距離センサを用いている点で透明物体等を把持することは考慮されていない。

\subsection{研究目的}
\label{sec:propose}
センサにおいて利点、欠点などの特徴がある。この中でも、私が着目したのは音である。
現在、最も頻繁に用いられているカメラと触覚センサのマルチモーダルなセンシングは、透明物体での近接での非接触センシングが必要な場合、弱点となることが多い。そこで、さらに、マイクを用いて物体にあたって反射した音を用いることができれば、より、ロバストに透明物体の検出をすることができると考えられる

そのため、本研究では音に着目する。しかし、現手法でよく用いられている音の到達時間差を用いる方法は空間分解能が低いため使用が難しい。そこで、光学においてより分解能が必要な場合に用いられるPhotometric Stereoのような直接的に法線の方向を取得する手法を参考にする。これにより、音波の時間差ではなく、波の強さに着目した、新しい音波を用いた法線方向の分類ではなく数値の推定手法の開発を目指す。


\subsection{本論文の構成}
\label{sec:organization}
本論文は全 6 章から構成される.以下に各章の概要を述べる.

\begin{itemize}
  \item \chapref{introduction}である本章は本研究の背景,関連研究と研究目的について述べた.
  \item \chapref{proposed_method}では,最も単純な形状である平面板の方位角の方向を取得する手法について2通り述べる。
  \item \chapref{exp_azimuth_angle}では,提案手法の有効性を示すための実験方法と実験結果、考察について述べる。
  \item \chapref{application_experiment}では,さらに、\chapref{exp_azimuth_angle}において見つかった疑問点について追加の検証実験について、方法と結果、考察を述べる。
  \item \chapref{app_hand}では,最終的にロボットで使用するためのセンサセットの作成方法、実験、結果、考察を述べる。
  \item \chapref{conclusion}では本研究の結論と今後の展望について述べる.
\end{itemize}

\newpage

% This page is written in English. This page is written in English. 

\newpage