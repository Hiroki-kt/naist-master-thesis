\section{緒言}
\label{chap:introduction}
\pagenumbering{arabic}

\subsection{研究背景}
\label{sec:back_ground}
家庭内には、カメラでの認識が難しいためにロボットにおいて把持が難しい物品が数多く存在する。例としては、コップやペットボトルといった透明な物体、鏡などの光が鏡面反射する物体等がある。更に、夜間などの光がない環境でも認識は難しい。そのため、夜間は人間と同じで明かりがないと動くことができない。このようなカメラが不得手とする物体の認識し、姿勢を認識する需要は増加している。特に、透明物体の把持については、コップやペットボトルといった身近なものにも存在するため、需要は大きい。

このような透明な物体の検出手法に関して研究は多数存在する。使用センサとしては、光学式距離センサ、カメラ、超音波センサ等、その種類も様々である。そのため、どの位置に透明な物体があるのかを認識することは可能である。

しかし、これをまた把持するとなるとまた、別の問題が出てくる。それは物体把持をする際に、位置というのは非常に重要であるが、位置だけでなく、同時に面の法線方向というのも非常に重要である。[参考文献 把持, 法線]

このように、透明な物体や鏡の法線方向推定はロボットにおける透明物体の把持の実現のためには避けて通れないものである。

\subsection{関連研究}
\label{sec:related_works}
ロボットによる透明物体の把持のための認識手法として、現状確立されたものは私が知る限り存在しない。カメラを用いる手法は非接触で認識可能であるため、一般的である。さらに、カメラと触覚センサのマルチモーダルセンシングの研究もある。これらの研究に加えて、透明物体ではないが、物体色にロバストな音波を用いての物体形状認識手法、物体の法線方向を推定する手法に関する研究を紹介する。

\subsubsection{カメラを用いての透明物体認識}
\label{sec:robot_pouring_human_skill}
例えば、コップやペットボトルといった透明な物体の認識を人間は目で行う。そのため、ロボットでもカメラを用いて透明物体を認識することはできると考えられている。しかし、現実はカメラでの透明な物体の認識は難しく、3DCADモデルを元にして、カメラを使い認識を行う研究がある。

しかし、透明物体の形状を3Dスキャンする技術は確立されておらず、まずどのようにして透明物体のCADモデルを作成するのか?といった問題がある。Ihrkeら[参考文献]は自分で1からCADモデルを作ることで、透明物体の認識を行った。しかし、この方法では非常に時間がかかる。次に、Phillipsら[参考文献]は、インターネットから似ている3DCADモデルを探してくることによって、より簡単に認識を行った。しかし、どうしても複雑な物体の認識ができなかった。最後にOsadchyら[参考文献]は透明なものにパウダーやスプレーを行うことによって、3Dスキャン用の透明じゃない物体を作成することによって、3Dモデルを作成した。この手法によって、様々な透明な物体の認識を行った。

\if 0
Yichaoら[]は、...
\fi

ロボットの把持への応用に関しては、Ilyaら[参考文献]は、Microsoft社のKinectを用いて透明な物体の認識を行った。透明な物体の3DCADモデルをスプレーで透明じゃなくすることで作成し、学習を行うことで、透明物体の物体姿勢の認識を行い、物体把持を行った。

\if 0
Ashutoshら[]は...

Luoら[]は...
\fi

しかし、これらのカメラを用いる場合には多くの場合、3DCADモデルを使用する必要があるため、既知の物体のみに対応しており、未知の物体の認識は難しい。

\subsubsection{カメラと触覚のマルチモーダルセンシング}
\label{sec:visual-based_pouring_motion}
カメラと触覚センサをマルチモーダルに使用する手法がの研究がされてきた。これらのマルチモーダルセンシングを使用することで未知の物体の把持も可能となる。

Kimら[参考文献]は、自作の触覚センサを用いて、柔らかい人間の指のようなマニピュレータを作成した。触覚センサのみで透明な物体でも把持することができる。しかし、触覚センサのみでは決められた位置の物体を掴むことはできても未知の位置にある物体を掴みたい際や、把持戦略を立てたい場合には難しい。

Shaoxiongら[]は、カメラと触覚センサをマルチモーダルに使用することによって未知の物体の3Dモデルを作成した。カメラのみでは認識が難しい透明なペットボトルの3Dモデルを作成することができた。しかし、特にペットボトルの精度は悪く、難しい点である。

このように、触覚センサを用いることによって、透明な物体でもロバストに認識ができるものの接触が必要であり、そこはカメラに頼らざるを得ないために大幅に結果が改善されるといったことはない。

\subsubsection{音を用いての物体認識手法}
\label{sec:tactile-based_manipulation}
次に、非接触で物体の色の影響を受けないとなると候補として上がるのが音である。音波は物体色に対してロバストであり、かつコウモリが使用するエコーロケーションに代表されるように物体の認識に関しての研究もある。

音波を用いての物体形状認識手法としてもっとも用いられる手法としては、到達時間差を用いての距離計測より物体形状を認識する手法が研究されている。
Olayaら[]は、水中における、ソナーを使用した環境認識を行った。水中の場合は空気中に比べて音速が早く、減衰も少ないため理想てきな環境ではあるが、空気中では同様の手法は用いることができない。
Davidら[]は,死角にある物体の形状をスピーカから出した音が反射して帰ってきたときの時間差を見ることによって、見えない位置にある物体の大体の形状を認識した。しかし、対象物体が大きくなければならないといった空間分解能の問題がある。

到達時間差を用いての計測には限界があると考えられる。そこで、比較的に小さい物体の形状認識手法として、用いられるのが反射音の大きさに着目した研究がある。

Amitら[×2]は、人間の手のジェスチャーを音を使って識別する手法の開発を行った。自作の超音波マイクアレイを用いて、スピーカから出た音が反射してマイクに入る音を取得することによって、ジェスチャーのクラス分類を行った。このときに、到達時間差だけでなく、音波の強さの情報を用いることによって、精度を向上させている。
この手法により、多少の物体形状は見ることができるが、まだ物体把持をすることができる程の分解能ではない。

Ohtaniら[]は、音波を用いて透明なペットボトルの大きさのクラス分類を行った。超音波マイクロホンアレイを用いる事によって、これらを実現したが、クラス分類までしかできておらず、物体把持には使えるものではないと考えれられる。

\subsubsection{物体法線方向取得手法}
これまで、透明物体の認識について、カメラ、触覚、音の三種類のセンサの視点から考えたが、上記の関連研究で物体把持まで行った研究は少ない。その理由としては、\secref{back_ground}でも述べたが、物体を把持する際には、物体位置だけでなく、法線方向も重要になる。
この法線方向推定手法について、これまでに提案された手法は,主に二通りである.一つは,距離センサを用いて,物体面上の任意の3点の位置を計測して、平面をあてはめる手法である.音波を用いる場合も,大まかな物体形状の認識は可能であるが、レーザーセンサや光学の距離センサを用いた場合と比較して分解能が低いため,正確な法線の取得は困難である.
二つ目はカメラを用いる場合である。この場合、ステレオカメラから認識する手法の他に、Photometric Stereoに代表される法線方向を直接推定する手法である.この場合、光の陰影を用いて推定されている。

それぞれの手法について、関連する研究を紹介する。

1つ目の距離センサを用いる場合、
Yamaguchiら[]は、光学式の距離センサアレイの開発を行い。このセンサを用いることで物体の法線方向を距離から認識し、物体把持を行う手法を提案した。さらに、反射光の波の強さから布と対象物体の識別をすることで、バックの中に入っている対象物体の取り出しが可能となった。しかし、欠点としては光学式のセンサアレイのために環境光の影響を受ける点や透明物体の把持はできない。

2つ目のカメラを用いる場合、
Pengら[]は、RGB-Dカメラを用いることによって、写真に写っている平面の法線方向の推定手法を提案した。さらに、この結果より3次元復元をを行っている。ものがいくつか置かれた状態であっても3次元復元が可能であった。

Guanyingら[]は、Photometric  Stereoの精度を上げるべく、Deep learning手法を付け加えることのよって、物体の色や環境光に対して、これまでの手法よりもロバストな手法を提案した。このPhotometric Stereoとは光源の方向が移動することによって、カメラにうつる物体の陰影は変化する。この陰影の変化について光の反射モデルを当てはめることによって、写真の1ピクセルの明るさとその点の法線方向に関係があるというShape-from-Shadingのアルゴリズムを応用した手法である。

1つ目の距離センサを用いる場合の手法に関しては、透明な物体に対してロバストな音を使った手法もある。しかし、2つ目の中でも特ににのPhotometric Stereoに代表される法線方向を取得する手法は透明物体や鏡に適応されることは少なく、さらに、特に光の陰影を用いる手法は、同じ波である音波を用いて法線方向を推定する手法は我々の知る限り存在しない.

\subsection{研究目的}
\label{sec:propose}
本研究では,Photometric Stereoのような直接的に法線の方向を取得する手法を音波に対して適用することで,音波を用いた法線方向取得手法の開発を目指し、様々な検証を行った。これによって、これまで問題となっていた音の空間分解能の低さを解決する可能性を検証を行うことで、将来的に透明物体や鏡といった物体の把持につなげていきたい。


\subsection{本論文の構成}
\label{sec:organization}
本論文は全 6 章から構成される.以下に各章の概要を述べる.

\begin{itemize}
  \item \chapref{introduction}である本章は本研究の背景,関連研究と研究目的について述べた.
  \item \chapref{proposed_method}では,最も単純な形状である平面板の方位角の方向を取得する手法について2通り述べる。
  \item \chapref{exp_azimuth_angle}では,提案手法の有効性を示すための実験方法と実験結果、考察について述べる。
  \item \chapref{application_experiment}では,さらに、\chapref{exp_azimuth_angle}において見つかった疑問点について追加の検証実験について、方法と結果、考察を述べる。
  \item \chapref{app_hand}では,最終的にロボットで使用するためのセンサーセットの作成方法、実験、結果、考察を述べる。
  \item \chapref{conclusion}では本研究の結論と今後の展望について述べる.
\end{itemize}

\newpage

% This page is written in English. This page is written in English. 

\newpage